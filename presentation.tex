% Auteur : Steve Prud’Homme
% Cette oeuvre, création, site ou texte est sous licence Creative Commons Attribution - Pas d’Utilisation Commerciale - Partage dans les Mêmes Conditions 4.0 International. Pour accéder à une copie de cette licence, merci de vous rendre à l'adresse suivante 
% http://creativecommons.org/licenses/by-nc-sa/4.0/ ou envoyez un courrier à 
% Creative Commons, 444 Castro Street, Suite 900, Mountain View, California, 94041, USA.

% :::SNIPET
% ::: SECTION
% \section{Contexte} 
%		\begin{frame}[allowframebreaks]
%			\frametitle{}
%			\begin {itemize}
%				\item 
%			\end{itemize}
%		\end{frame}


\documentclass{beamer}
\usepackage{color}
\usepackage{beamerthemesplit} % new 
\usepackage[french]{babel}
\usepackage[utf8]{inputenc}
\usepackage{tikz}
\usepackage[fixlanguage]{babelbib}
\selectbiblanguage{french}
% Natlib pour la bibliographie
\usepackage{natbib}
\usepackage{url}
\usetikzlibrary{mindmap,shadows,shapes,backgrounds}
\usepackage[T1]{fontenc}
\setbeamertemplate{bibliography item}[text]
\usepackage{multicol}


\definecolor{MightySlate}{RGB}{85,98,112}
\definecolor{Pacifica}{RGB}{78,205,196}
\definecolor{AppleChic}{RGB}{199,244,100}
\definecolor{CheeryPink}{RGB}{255,107,107}

\setbeamercolor{titlelike}{parent=structure}
\setbeamerfont*{title}{size=\huge}
\setbeamercolor{title}{bg=MightySlate, fg=white}
\setbeamercolor{author}{bg=Pacifica, fg=white}
\setbeamercolor{institute}{bg=AppleChic, fg=black}
\setbeamercolor{date}{bg=CheeryPink, fg=white}

\definecolor{DTUred}{RGB}{178,20,20}
\setbeamercolor*{palette primary}{use=structure,fg=white,bg=MightySlate}
\usepackage{helvet}
\usepackage{draftwatermark}
\SetWatermarkLightness{0.5}
\SetWatermarkAngle{25}
\SetWatermarkScale{0.5}
\SetWatermarkFontSize{2cm}
\SetWatermarkText{Document de travail}

\begin{document}
	\title{Survol des situations de travail, des processus de production, du contrôle de la qualité et des bonnes pratiques en conception et réalisation d’outils pédagogiques en ligne} 
	\author{Steve Prud'Homme} 
	\institute{GTN-Québec - Commission scolaire de Laval} 
	\date{\today} 

	
	%\usebackgroundtemplate{%
  %\includegraphics[width=\paperwidth,height=\paperheight]{sommaire.png}} 
	
	\frame{\titlepage} 
	\usebackgroundtemplate{ } 
	\section{Sommaire} 
		\begin{frame}
			Cette présentation vise à :
			\frametitle{Sommaire}
			\begin {itemize}
				\item Effecturer un court \textbf{état des lieux} de l'évaluation en ligne et de ses pratiques en formation professionnelle
				\item \textbf{Familiariser} l'auditoire avec l'évaluation en ligne en présentant un bref aperçu de la \textbf{littérature} québécoise, étatsunienne et britanique sur le sujet
				\item Démontrer qu'il est pertinent d'adopter des \textbf{pratiques harmonisées} en ce qui concerne l'évaluation en ligne
				\item Promouvoir l'idée qu'il est essentiel que le MEES \textbf{établisse des lignes directrices} rapidement en ce qui concerne l'évaluation en ligne

			\end{itemize}
		\end{frame}
	\frame[allowframebreaks]{\frametitle{Ordre du jour}\tableofcontents}


	\section{Contexte} 
		\begin{frame}[allowframebreaks]
			\frametitle{}
			\begin {itemize}
				\item 
			\end{itemize}
		\end{frame}
		
		


\section{Bibliographie}
\subsection{Bibliographie}
\frame[allowframebreaks]{\frametitle{Bibliographie}

\bibliographystyle{apalike}
\bibliography{bibliographie} %bibtex file name without .bib extension
}
\end{document}

